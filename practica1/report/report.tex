\documentclass[11pt]{article}
\renewcommand{\baselinestretch}{1.05}
\usepackage[spanish]{babel}
\usepackage[utf8]{inputenc}
\usepackage{lipsum}

\usepackage{amsmath,amsthm,verbatim,amssymb,amsfonts,amscd}
\usepackage{graphicx, wrapfig}
\usepackage{float}
\usepackage{caption, subcaption}
\usepackage{tkz-fct}
\usetikzlibrary{babel}
\usepackage{pgfplots}
\usepackage{enumitem}
\usepackage{multicol, vwcol}
\usepackage{listingsutf8}
\usepackage{color}
\usepackage{hyperref}
\usepackage{booktabs}
\usepackage{biblatex}
\bibliography{bibliography.bib}
\definecolor{lightgray}{gray}{0.95}

\hypersetup{
    bookmarks=true,         % show bookmarks bar?
    unicode=false,          % non-Latin characters in Acrobat’s bookmarks
    pdftoolbar=true,        % show Acrobat’s toolbar?
    pdfmenubar=true,        % show Acrobat’s menu?
    pdffitwindow=false,     % window fit to page when opened
    pdfstartview={FitH},    % fits the width of the page to the window
    pdftitle={Programación web - Práctica 1},    % title
    pdfauthor={Francisco Luque},     % author
    pdfsubject={Programación web},   % subject of the document
    pdfnewwindow=true,      % links in new PDF window
    colorlinks=true,        % false: boxed links; true: colored links
    linkcolor=red,          % color of internal links (change box color with linkbordercolor)
    citecolor=cyan,         % color of links to bibliography
    filecolor=magenta,      % color of file links
    urlcolor=blue           % color of external links
}

\setlength{\parindent}{0pt}
\topmargin0.0cm
\headheight0.0cm
\headsep0.0cm
\oddsidemargin0.0cm
\textheight23.0cm
\textwidth16.5cm
\footskip1.0cm
\theoremstyle{plain}

\newtheorem{theorem}{Teorema}
\newtheorem{corollary}{Corolario}
\newtheorem{lemma}{Lema}
\newtheorem{proposition}{Proposición}
\theoremstyle{definition}
\newtheorem{definition}{Definición}
\newtheorem{example}{Ejemplo}

\newcommand{\N}{\mathbb{N}}
\newcommand{\Z}{\mathbb{Z}}
\newcommand{\Q}{\mathbb{Q}}
\newcommand{\C}{\mathbb{C}}
\newcommand{\R}{\mathbb{R}}

\begin{document}

\title{Programación Web - Práctica 1\\
  DGIIM \\
  \large Creación de una página web utilizando HTML y CSS}
\author{Francisco Luque Sánchez}
\maketitle

\begin{figure}[H]
  \centering
  \includegraphics[width=.3\textwidth]{html_css.png}
\end{figure}


\section{Introducción}

En esta práctica se ha desarrollado una página web para un gimnasio
utilizando HTML y CSS. Se ha tratado de crear una web responsive, de
forma que se vea correctamente en pantallas con distinta proporción
(concretamente, en pantallas de proporción 4:3 y 16:9, a una
resolución de $1024 \times 768$ y $1920 \times 1080$ respectivamente)
y dispositivos móviles.\\

Pasamos a comentar con mayor profundidad el desarrollo de dicha web.

\section{Estructura HTML de la página}

La página se ha estructurado en las siguientes secciones:

\begin{itemize}
\item Inicio (\texttt{index.html}): Es la página a la que se accede
  por primera vez, cuando se accede al sitio web. En esta página no se
  puede hacer nada, y los enlaces que hay en ella redirigen de nuevo a
  esa página principal. Este es el comportamiento que se busca en la
  página para los usuarios que no están identificados. En dicha página
  se permite únicamente que el usuario se identifique. El sistema de
  identificación está simulado, ya que ahora mismo no tenemos las
  herramientas para gestionar un sistema de identificación (un
  lenguaje como PHP).
\item Inicio identificado (\texttt{index2.html}): Es la página a la
  que accede el usuario después de identificarse en el sistema. En
  futuras prácticas, esta página dejará de tener sentido, ya que con
  el index original será suficiente. Los enlaces se activarán en dicho
  index cuando el usuario esté identificado. En esta práctica, el
  \texttt{index2.html} es una página idéntica al \texttt{index.html},
  pero en la que el usuario aparece identificado, y los enlaces al
  resto de páginas están activos.
\item Actividades (\texttt{actividades.html}): Página en la que se ve
  la información sobre las actividades que se llevan a cabo en el
  gimnasio.  Además, aparece información sobre noticias que tienen que
  ver con el gimnasio.  Dentro de la página de actividades tenemos
  enlaces a cada una de las actividades citadas, en la que podemos ver
  información detallada sobre la misma, como su descripción y el
  horario.
\item Horario (\texttt{horario.html}): En esta página se puede
  consultar el horario del gimnasio.
\item Técnicos (\texttt{tecnicos.html}): Página que muestra el
  personal que trabaja en el centro.
\item Servicios (\texttt{servicios.html}): Página que muestra los
  servicios ofertados por el gimasio, así como sus instalaciones.
\item Localización (\texttt{localizacion.html}): Página en la que se
  muestra la localización del centro, utilizando un mapa creado con
  \textit{Google Maps}.
\item Precios (\texttt{precios.html}): Página que permite consultar
  los distintos planes ofertados por el gimnasio para sus clientes.
\item Alta usuarios (\texttt{altausuario.html}): Página que tiene un
  formulario que permite a una persona crearse un nuevo usuario en el
  sistema. Dicho formulario no tiene ninguna funcionalidad
  actualmente, ya que como hemos comentado anteriormente no disponemos
  de un lenguaje de programación que nos permita gestionar los datos
  introducidos. Esta funcionalidad se implementará en futuras
  prácticas.
\item Foro (\texttt{foro.html}): En esta página se ha simulado el
  funcionamiento de un foro. En ella se observan los hilos creados por
  distintos usuarios, y se permite acceder a dos nuevas páginas:
  \begin{itemize}
  \item Crear hilo (\texttt{crearhilo.html}): Aquí se muestra un
    formulario que permite la creación de un hilo nuevo, introduciendo
    su título y la descripción
  \item Responder hilo (\texttt{responderhilo.html}): Este formulario
    permite responder a un hilo ya existente.
  \end{itemize}
  Al igual que los anteriores, estos formularios carecen de
  funcionalidad actualmente, funcionalidad que se implementará más
  adelante.
\end{itemize}

Una vez explicada la estructura de archivos HTML, vamos a ver cómo está
organizado el CSS de la página.

\section{Archivos CSS de la página}

Pasamos a ver los archivos CSS que se han utilizado en la página. Se
han creado dos conjuntos de archivos distintos, unos que se cargan
cuando la página web se está consultando desde un ordenador (la
anchura de la página es mayor a 768 píxeles) y otros que se cargan
cuando se accede a la página desde ún móvil (la anchura del
dispositivo es menor a 768 píxeles). Para un desarrollo realmente
completo se habría tenido que especificar la configuración para otros
tamaños, pero dado que lo que se pretende con la práctica es aprender
el funcionamiento básico de CSS no se ha profundizado más en este
apartado. En realidad, si lo que se busca es crear un página
completamente responsive, crear el CSS desde cero es una tarea muy
costosa, y probablemente, teniendo en cuenta la cantidad de
dispositivos distintos que existen actualmente, cada uno con sus
dimensiones particulares, sería prácticamente imposible que la
visualización fuera perfecta en todos ellos. Usualmente, lo que se
hace en estos casos es utilizar librerías ya creadas que se preocupan
en que el diseño de las páginas web sea completamente responsive. De
esta forma, el desarrollador web no tiene que preocuparse de que la
página se visualice correctamente en todos los dispositivos, sino que
es la propia librería la que le aporta las herramientas para
conseguirlo de forma cómoda. Un ejemplo de estas librerías,
probablemente la más extendida, es \textit{Bootstrap}
\cite{bootstrap}, desarrollada por el equipo de \textit{Twitter}.

Veamos a continuación los diferentes archivos CSS que se utilizan en
la página web:

\begin{itemize}
\item \texttt{topbar.css}: En este CSS se establece el estilo de la
  barra superior de la página, en la que se muestra el logo de la
  misma, el nombre del gimnasio, y la información sobre la
  autenticación del usuario.
\item \texttt{top-menu.css}: En este CSS se establece el estilo del
  menú superior, que aparece en todas las páginas excepto en la
  principal, y es la que nos permite movernos por todo el sitio
  web. Una particularidad de este menú es que se ha conseguido que al
  descender por la página, el mismo ocupe siempre la parte superior,
  de forma que siempre sea visible.  Así, aunque un usuario se
  encuentre en la parte inferior de una página, podrá dirigirse a otra
  página sin tener que volver a subir. Esto se consigue con una
  combinación de la propiedad \texttt{position: sticky}, que permite
  desplazar un elemento con el resto de la página cuando al hacer
  \texttt{scroll} debería quedarse oculto, y la propiedad
  \texttt{z-index}, que permite que un elemento quede por encima del
  resto cuando se superponen.
\item \texttt{footer.css}: En este CSS se establece el estilo del pie
  de página. Este componente se puede observar en la parte inferior de
  todas las páginas. Se han desarrollado dos clases distintas para el
  pie. Una de ellas se mantiene en la parte inferior de la página
  independientemete del tamaño que ocupe el contenido de la misma, que
  se usa en las páginas en las que el contenido no rellena
  completamente la pantalla, como la página de inicio o la página de
  actividades. La otra coloca el footer debajo de todo el
  contenido. Esto se ha hecho para que, en las páginas que hay poco
  contenido, se rellene la pantalla completamente, a fin de no romper
  la estética de la misma.
\item Archivos \texttt{nombre-pagina.css}: Por cada página HTML que
  hay creada, se tiene además un archivo CSS en el que se especifica
  el estilo del contenido que hay en la página. Se ha optado por
  organizar el CSS de esta forma porque cada página tiene un contenido
  específico y distinto de las demás. De esta manera, la estructura de
  archivos permite encontrar el CSS referente a cada página de forma
  más sencilla, por si hay que realizar alguna modificación. Se podría
  haber optado, en algunos casos, por utilizar la misma hoja de
  estilos para más de una página, como es el caso de las páginas
  \texttt{servicios.html} y \texttt{tecnicos.html}, en los que la
  estructura que siguen ambas es muy similar. No obstante, se ha
  optado finalmente por la organización separada, para poder controlar
  de forma independiente ambas páginas en caso de que fuera necesario.
\end{itemize}

Los archivos indicados anteriormente se encuentran en el directorio
\texttt{static/css}. Además de estos archivos, dentro de este
directorio se encuentra la carpeta \texttt{mobile}. En ella se
encuentran los mismos archivos indicados anteriormente, pero en la
versión móvil. De nuevo, se podría haber tenido un solo archivo con el
CSS para cada página, y que las \textit{media queries} se realizasen
dentro del propio CSS, pero buscando una organización más modular del
código, dicha \textit{media query} se hace en el HTML, y se carga un
CSS u otro dependiendo del tamaño del dispositivo que acceda a la
página.\\

En la versión móvil, lo que se ha intentado, es colocar los elementos
que aparecen alineados horizontalmente en la versión de escritorio
colocados en vertical. Esto se ha hecho porque, usualmente, los usuarios
que acceden a una web desde un móvil lo hacen con el móvil colocado
en vertical, al contrario que pasa con las pantallas de escritorio,
que tienen una disposición apaisada. De esta forma, en la versión
de escritorio en la pantalla de actividades aparece el listado de
actividades a la izquierda y el de noticias a la derecha, mientras
que en la versión móvil aparecen arriba las actividades y debajo las
noticias.

\section{Aspectos de implementación}

El desarrollo de la práctica se ha realizado utilizando las
herramientas de desarrollador que aporta \textit{Google Chrome}. Estas
herramientas permiten cambiar, de forma cómoda, el tamaño de la
pantalla en la que se mostraría la página. Esto permite visualizar de
forma sencilla la página en distintos dispositivos, sin necesidad de
tener los mismos físicamente. Es con esta herramienta con la que se ha
trabajado para tratar de hacer una página \textit{responsive}.  Se ha
probado la misma en todas las opciones que el navegador incluye por
defecto, y se ha comprobado que la página se comporta bien en todas
ellas. De esta forma podemos garantizar, de forma más o menos segura,
que vamos a poder visualizarla correctamente en cualquier dispositivo.
No obstante, la visualización perfecta en todos ellos es difícil de
garantizar completamente al desarrollar el CSS completamente desde
cero.\\

En cuanto al comportamiento responsive de la página, se ha seguido una
filosofía similar a la que sigue \textit{Bootstrap} en su desarrollo.
Lo que se hace es tener en cuenta el ancho de la pantalla para
redimensionar los elementos, mientras que en vertical se mantienen
los tamaños independientemente de la altura del dispositivo. Esto se
hace porque suele ser más natural el \textit{scroll} vertical que
el horizontal.


\section{Bibliografía}

\printbibliography

\end{document}
