\documentclass[11pt]{article}
\renewcommand{\baselinestretch}{1.05}
\usepackage[spanish]{babel}
\usepackage[utf8]{inputenc}
\usepackage{lipsum}

\usepackage{amsmath,amsthm,verbatim,amssymb,amsfonts,amscd}
\usepackage{graphicx, wrapfig}
\usepackage{float}
\usepackage{caption, subcaption}
\usepackage{tkz-fct}
\usetikzlibrary{babel}
\usepackage{pgfplots}
\usepackage{enumitem}
\usepackage{multicol, vwcol}
\usepackage{listingsutf8}
\usepackage{color}
\usepackage{hyperref}
\usepackage{booktabs}
\usepackage[sorting=none]{biblatex}
\bibliography{bibliography.bib}

\definecolor{lightgray}{gray}{0.95}

\hypersetup{
    bookmarks=true,         % show bookmarks bar?
    unicode=false,          % non-Latin characters in Acrobat’s bookmarks
    pdftoolbar=true,        % show Acrobat’s toolbar?
    pdfmenubar=true,        % show Acrobat’s menu?
    pdffitwindow=false,     % window fit to page when opened
    pdfstartview={FitH},    % fits the width of the page to the window
    pdftitle={Programación web - Práctica 2},    % title
    pdfauthor={Francisco Luque},     % author
    pdfsubject={Programación web},   % subject of the document
    pdfnewwindow=true,      % links in new PDF window
    colorlinks=true,        % false: boxed links; true: colored links
    linkcolor=red,          % color of internal links (change box color with linkbordercolor)
    citecolor=cyan,         % color of links to bibliography
    filecolor=magenta,      % color of file links
    urlcolor=blue           % color of external links
}

\setlength{\parindent}{0pt}
\topmargin0.0cm
\headheight0.0cm
\headsep0.0cm
\oddsidemargin0.0cm
\textheight23.0cm
\textwidth16.5cm
\footskip1.0cm
\theoremstyle{plain}

\newtheorem{theorem}{Teorema}
\newtheorem{corollary}{Corolario}
\newtheorem{lemma}{Lema}
\newtheorem{proposition}{Proposición}
\theoremstyle{definition}
\newtheorem{definition}{Definición}
\newtheorem{example}{Ejemplo}

\newcommand{\N}{\mathbb{N}}
\newcommand{\Z}{\mathbb{Z}}
\newcommand{\Q}{\mathbb{Q}}
\newcommand{\C}{\mathbb{C}}
\newcommand{\R}{\mathbb{R}}

\begin{document}

\title{Programación Web - Práctica 2\\
  DGIIM \\
  \large Web con contenido dinámico usando Javascript y PHP}
\author{Francisco Luque Sánchez}
\maketitle

\begin{figure}[H]
  \centering
  \includegraphics[width=.5\textwidth]{js_php.png}
\end{figure}


\section{Introducción}

En esta práctica se ha dotado de contenido dinámico a la página web
que se desarrolló en la práctica anterior, utilizando para ello PHP
como lenguaje de \textit{scripting} en la lado del servidor, y
JavaScript (concretamente la librería JQuery) como lenguaje
de \textit{scripting} en el lado del cliente.\\

Pasamos a comentar el desarrollo completo en más profundidad.

\section{Código en PHP de lado del servidor}

El código desarrollado para dotar a la página web de contenido
dinámico del lado del servidor se ha hecho utilizando PHP. Con este
lenguaje se gestionan todas las consultas a la base de datos, así como
la inclusión de contenido redundante en todos los documentos HTML que
conforman la página. Todos los archivos con código PHP puro que se
han desarrollado se encuentran dentro de la carpeta
\texttt{php-includes}. Comentaremos brevemente cómo se han
estructurado los \textit{scripts} PHP dependiendo de la función que
realizan.

\subsection{Archivos PHP para incluir contenido redundante}

Como hemos dicho en la introducción, se ha usado este lenguaje para
incluir en los documentos HTML el código redundante que aparecía en
todos ellos. Los \textit{scripts} que se han desarrollado destinados a
esta funcionalidad son los siguientes:

\begin{itemize}
\item \texttt{generic-head.inc.php}: Este archivo se encarga de
  imprimir una cabecera genérica para todos los archivos HTML. En
  todos los documentos HTML que conforman la página se incluían al
  principio la codificación que utilizaba la página (UTF-8), el nombre
  de la misma, el autor, y los archivos CSS que especificaban el
  estilo que tenían el menú superior y el pie de página. Para evitar
  tener que modificar esta información en todos ellos cada vez que se
  produjese un cambio, se ha creado este script que introduce toda
  esta información en dichos documentos. Además, se utiliza este
  script para introducir el script de JavaScript que define JQuery, ya
  que todas las páginas con contenido dinámico lo necesitan. Los
  archivos CSS específicos de cada página, no obstante, se incorporan
  al código HTML de la misma manera que se hacía en la práctica 1.
\item \texttt{header.inc.php}: Este archivo se encarga de imprimir
  todo el código que se incluye en el menú superior de las páginas. De
  la misma forma que en caso anterior, todas las páginas cuentan con
  el menú superior en el que aparece el logo del gimnasio, el nombre
  del mismo, y el formulario de inicio de sesión. Para poder tener
  este menú junto con su funcionalidad en todas las páginas, se
  imprime toda la información necesaria para que funcione el mismo
  desde este archivo, tanto el código HTML como el script de
  JavaScript que gestiona el formulario de login.
\item\texttt{top-menu.inc.php}: Este archivo imprime todo el código
  HTML que define el menú de navegación que aparece en todas las
  páginas del sitio excepto la página principal (\texttt{index.php}).
\item \texttt{footer.inc.php}: Este archivo imprime el pie de página
  de la misma forma que lo hacían los anteriores. La única diferencia
  que tiene este archivo con el resto es que en lugar de imprimir
  directamente el código HTML, tiene una función que recibe un
  parámetro, correspondiente a la clase que se asigna al footer. Se ha
  hecho así porque no todas las páginas tienen el mismo footer, si no
  que éste tiene un comportamiento distinto en las páginas cuyo
  contenido ocupa más que la ventana y en las que esto no ocurre. De
  esta forma, con una misma función hemos conseguido manejar estos
  dos comportamientos, simplemente con el parámetro de la función que
  imprime el contenido del footer.
\end{itemize}

\subsection{Archivos PHP para el manejo de la base de datos}

El resto de archivos desarrollados se encargan del manejo de la base
de datos en el servidor. Se ha intentado estructurar el código en
clases, cada una de ellas representando cada una de las tablas de la
base de datos que se han creado para la práctica. Vamos
a describir más en profundidad dichas clases.\\

Comenzamos hablando del archivo \texttt{db-config.inc.php}. Este
archivo no implementa ninguna funcionalidad, sino que define las
constantes necesarias para el funcionamiento del resto de clases.  En
este archivo se incluye, por tanto, una constante que contiene la
información necesaria para la conexión a la base de datos (DSN), el
nombre de usuario y la contraseña del usuario de la base de datos, y
constantes con el nombre de las tablas creadas
(users, forum\_threads y forum\_responses).\\

El siguiente archivo a mencionar es
\texttt{database-management.inc.php}.  Este archivo define una clase
abstracta que representa una tabla en la base de datos. Los métodos
implementados en esta clase son los siguientes:

\begin{itemize}
\item \texttt{\_\_construct}: Este método es el constructor de la
  clase.  Recibe como parámetro un array y rellena otro array
  (atributo de la clase) con la información contenida en el array que
  se pasa como argumento.
\item \texttt{getValue}: Esta función permite recuperar contenido del
  array atributo de la clase, especificando el nombre que lo
  identifica.
\item \texttt{connect}: Este método permite abrir una conexión con la
  base de datos.
\item \texttt{disconnect}: Este método cierra una conexión activa con
  la base de datos.
\end{itemize}

El resto de clases que se describirán a continuación heredan de la
clase comentada. Comentaremos además el resto de métodos que se
han añadido a cada una.\\

Comenzamos comentando la clase \texttt{User}, la cual se encuentra
implementada en el archivo \texttt{user.inc.php}. Esta clase
representa un usuario almacenado en la base de datos del sistema. Por
este motivo, su array de datos está preparado para contener toda la
información que se guarda de un usuario. Tiene implementados los
siguientes métodos:

\begin{itemize}
\item \texttt{saveUser}: Este método guarda el usuario que llama al
  mismo en la base de datos. Abre una conexión utilizando la función
  de la clase que hereda, realiza la inserción, y cierra la conexión
  al finalizar. Esta función se utiliza a la hora de dar de alta a un
  usuario nuevo.
\item \texttt{logUser}: Este método intenta permitir el acceso a un
  usuario al sistema. Toma el nombre de usuario y la contraseña del
  array de datos del que dispone, y trata de buscar en la base de
  datos un usuario para el que coincida esa información. Si lo
  encuentra, inicia una sesión de PHP y añade al array de sesión una
  entrada nueva con el nombre de usuario. Si no encuentra al usuario,
  devuelve un mensaje diciendo que no ha sido posible identificar al
  usuario en la base de datos. Esta función se utiliza cuando se
  intenta acceder al sistema con el menú de inicio de sesión que
  aparece en la parte superior de todas las páginas.
\item \texttt{getUser}: Esta función de clase devuelve un usuario
  almacenado en la base de datos dado su nombre de usuario como
  argumento. Se utiliza para recuperar la información de un usuario
  cuando se accede a la pestaña de perfil.
\item \texttt{updateUser}: Esta función actualiza la información de un
  usuario en la base de datos cuando se guarda su perfil tras
  modificarlo. El funcionamiento es muy similar al de
  \texttt{saveUser}, con la salvedad de que ahora la sentencia es de
  tipo \texttt{UPDATE}, en lugar de \texttt{INSERT}.
\end{itemize}

Pasamos ahora a comentar la clase \texttt{ForumThread}, que representa
un hilo del foro. Está implementada en el archivo
\texttt{forum-thread.inc.php}. Esta clase guarda en el array de datos
algunas entradas que no pertenecen estrictamente a la tabla
\texttt{forum\_threads} de la base de datos. Concretamente, además de
la información del hilo, guarda el nombre completo y la ruta a la
imagen del usuario creador del hilo. Se hace esto para simplificar la
recolección de información a la hora de mostrar los hilos en el
foro. Los métodos implementados en esta clase son los siguientes:

\begin{itemize}
\item \texttt{createThread}: Esta función guarda en la base de datos
  el hilo que realiza la llamada a la misma. Al igual que ocurría con
  la función \texttt{saveUser}, esta función abre una conexión a la
  base de datos usando la función de la clase padre, realiza la
  consulta de tipo \texttt{INSERT}, y cierra la conexión.
\item \texttt{getThreads}: Función de clase que devuelve todas las
  hebras que hay en la base de datos. Se utiliza esta función para
  crear la vista principal del foro.
\item \texttt{getThread}: Esta función de clase recibe como parámetro
  el identificador de un hilo y devuelve un objeto de la clase
  \texttt{ForumThread} con toda la información referente al mismo.  Se
  utiliza esta función para mostrar la información de un hilo
  específico en su página detalle.
\item \texttt{getThreadsFromUser}: Esta función de clase recibe como
  argumento un identificador de usuario y devuelve todos los hilos del
  foro escritos por este usuario, junto con sus identificadores. Se
  utiliza esta función para recuperar todos los hilos de un usuario y
  mostrarlos cuando se pasa el ratón por encima de su imagen en el
  foro.
\end{itemize}

La última clase implementada es \texttt{ForumResponse}, que representa
una respuesta a un hilo del foro. Se encuentra en el archivo
\texttt{forum-response.inc.php} Esta clase, al igual que la anterior,
guarda información sobre una respuesta, así como el nombre y la ruta a
la imagen del usuario que ha registrado la respuesta. Esta clase sólo
cuenta con dos métodos, el método \texttt{createResponse}, que guarda
la respuesta en la base de datos, y la función de clase
\texttt{getResponses}, que dado el identificador de un hilo devuelve
un array con los objetos de la clase \texttt{ForumResponse} que
representan todas las respuestas que se han dado a ese hilo. Este
método se utiliza a la hora de recuperar todas las respuestas a un
hilo para mostrarlas en la página de detalle del mismo.\\

Se explican a continuación los scripts que gestionan los distintos
formularios que aparecen en la página web. Comenzamos comentando el
archivo \texttt{log\_user.php}. Este archivo gestiona la
identificación de usuarios del lado del servidor. Este archivo recoge
por POST la información que se ha introducido en el cuadro de
identificación de usuario que aparece en el menú superior de la
página, crea un usuario con el nombre de usuario y la contraseña
especificados, y llama a la función \texttt{logUser} de la clase
\texttt{User} explicada anteriormente. Como ya dijimos, esta función
devuelve una cadena de texto que indica si el usuario ha sido
correctamente identificado en el sistema (el nombre de usuario y
contraseña especificados corresponden a un usuario registrado en la
base de datos) o si no ha sido posible encontrar dicho
usuario. Dependiendo de la respuesta que de dicha función, un script
de JavaScript que comentaremos más adelante se encargará de avisar al
usuario de que no se ha introducido
la información correcta o dará al usuario por identificado.\\

El siguiente archivo que maneja formularios es \texttt{logout.php}.
Se llama a este archivo cuando se pulsa el botón de salir del menú
superior, el cual se muestra cuando un usuario está identificado y con
una sesión abierta. Este archvo lo único que hace es abrir la sesión,
eliminar el elemento ``usr'' del array de sesión, y redirigir a la
página principal.\\

Otro de los archivos que gestiona formularios es
\texttt{sign\_up.php}.  Este archivo se encarga de dar de alta a un
nuevo usuario en el sistema. Recoge por POST toda la información de
usuario introducida, crea un usuario con dicha información, y llama a
la función \texttt{saveUser}. Además, coge la foto subida por el
usuario del array FILES y la coloca en la carpeta
\texttt{/static/imgs/users}
del servidor.\\

El siguiente archivo de gestión de formularios es
\texttt{update\_user.php}. Este tiene un funcionamiento muy similar al
anterior. Recoge por POST la información del usuario, crea un objeto
de la clase \texttt{User}, y llama a la función \texttt{updateUser}.

Pasamos ahora a comentar los archivos que gestionan los formularios
del foro. Tenemos dos archivos dedicados a esta tarea. Uno de ellos es
\texttt{create-thread.php}, que se encarga de crear un hilo nuevo.
Este archivo recoge por POST el título y el cuerpo del hilo, y del
array de sesión el nombre del usuario que está identificado, crea un
hilo con esa información y lo guarda en la base de datos. El otro
archivo es \texttt{response-thread.php}, que registra una respuesta a
un hilo, cogiendo por POST el cuerpo de la respuesta y el ID del hilo
que se quiere responder, y del array de sesión el nombre de usuario
que registra la respuesta.

\subsection{Código en PHP para mostrar contenido en los documentos
  HTML}

Además del código PHP que se encuentra en los archivos mencionados
anteriormente, hay pequeños fragmentos de código en PHP dentro de los
archivos HTML de la página, que permiten mostrar un contenido u otro
dependiendo de la información que está guardada en la base de
datos, así como de si hay un usuario con una sesión abierta o no.\\

Lo primero que comentaremos es el comportamiento de la página en
función de si hay o no un usuario con una sesión abierta. Hay tres
partes de la página que modifican su comportamiento en este caso.  Por
un lado, en el menú superior, si no hay un usuario identificado se
muestra el formulario de inicio de sesión. Si dicha sesión sí que
existe, se muestra el nombre del usuario que hay identificado junto
con un botón para cerrar sesión. Otra parte en la que el
comportamiento es distinto es en el foro. Al contrario que en la
práctica anterior, ahora se permite que una persona sin identificar
navegue por la página. Esto se hace porque no tiene mucho sentido que
una persona sin identificar esté obligada a crear un usuario para
recibir información sobre el centro deportivo. No obstante, no se
quiere que un usuario no identificado pueda añadir nuevos hilos al
foro o responder a los ya existentes. Teniendo esta idea en mente, lo
que se hace es que si no hay un usuario con una sesión abierta, en el
foro no se muestran los botones ni los formularios que permiten crear
nuevos hilos ni responder a los que ya existen. Finalmente, en los
menús de la página no se muestra la pestaña de perfil si el usuario no
está identificado. Esto se debe a que esa pestaña está pensada para
que un usuario del sistema modifique su información. Si no hay un
usuario identificado, no hay usuario al que modificar, y por tanto
esta pestaña es inútil.\\

Ahora, como última funcionalidad de la página escrita en PHP se tiene
el mostrar contenido de forma dinámica en función de lo que hay
guardado en la base de datos. Existe funcionalidad de este tipo en dos
partes de la página. Por un lado, en el perfil del usuario, se utiliza
la información del usuario con sesión iniciada que está guardada en la
base de datos para rellenar el formulario de cambio de información con
los datos actuales del mismo. Por otro lado, en el foro, se utilizan
los métodos de las clases \texttt{ForumThread} y
\texttt{ForumResponse} para mostrar toda la información de hilos y
respuestas que está almacenada en la base de datos.\\

Una vez visto todo el código en PHP que se ha desarrollado para la
página web, pasamos a ver el código en JavaScript.

\section{Código en JavaScript de lado del cliente}

El código que se ha desarrollado en JavaScript se utiliza
principalmente para la validación de formularios, así como para
mostrar en el foro la información relativa a los hilos creados por un
usuario. Todo el código en JavaScript se encuentra en la carpeta
\texttt{/static/js}. El primer archivo que vamos a comentar es
\texttt{jquery-3.3.1.min.js}. Este archivo es el correspondiente a la
librería JQuery, que es la que se ha usado para desarrollar el
código. Se ha optado por usar esta librería por varias cosas. La
primera de ellas, es la posibilidad de usar AJAX de una forma mucho
más simple que con JavaScript puro. JQuery proporciona una función
llamada \texttt{ajax()}, la cual recibe como parámetro un diccionario
con el nombre del script PHP al que se envía la información, la
información que se quiere enviar vía POST, y opcionalmente el
comportamiento que se quiere provocar en caso de que la petición tenga
éxito o falle. Otra de las ventajas del uso de JQuery es la
simplicidad que tiene para seleccionar elementos del DOM. En lugar de
tener que usar las funciones \texttt{document.getElementById()},
\texttt{document.getElementsByClassName()}..., en JQuery se unifica
todo este comportamiento con lo que se llaman selectores
(\texttt{Selectors}). Permiten la selección de forma muy simple
mediante la sintaxis \texttt{\$(<selector>)}, donde \texttt{selector}
es una cadena de texto que identifica al elemento que queremos
seleccionar.  Este selector puede hacer referencia al id del objeto, a
su clase, a su nombre... Finalmente, el sistema de eventos del que se
dispone es muy potente, para el que se usa la sintaxis
\texttt{\$(<selector>).<event>( function () \{...\} )}. Esto permite
asociar una función a un objeto (el seleccionado por
\texttt{selector}), la cual se realiza cuando tiene lugar el evento
indicado por \texttt{<event>}. De esta manera, podremos asociar
funciones a los distintos eventos que tengan lugar en la página, como
el envío de un formulario (evento \texttt{submit}), o cuando se pase
el ratón
por encima de una imagen en el foro (evento \texttt{hover}).\\

Pasamos a comentar los scripts de JavaScript que han sido
implementados para la página. El primer script que comentaremos es
\texttt{manageLogin.js}. Este script se encarga de manejar el evento
que se produce cuando se manda el formulario de identificación de
usuario. Este script recoge los datos del formulario (usuario y
contraseña introducidos) y manda una petición AJAX al archivo PHP
\texttt{log\_user.php}. Una vez terminada la petición AJAX, si la
misma ha sido realizada con éxito, comprueba la respuesta que ha
mandado el script PHP. Si la misma ha sido satisfactoria, es decir, si
el usuario se ha identificado correctamente, se recarga la página en
la que se encontrase el usuario. En caso contrario, se muestra una
alerta indicando que el usuario y la contraseña introducidos no son
correctos, sin necesidad de recargar la página. La ventaja de hacer
esto es, por un lado, mejorar la experiencia del usuario no
introduciendo esperas innecesarias, y por otro, reducir la carga del
servidor, al no tener que enviar una petición para recargar la
página.\\

El siguiente archivo que comentaremos es \texttt{signUp.js}. Este
archivo se encarga de validar el formulario de alta de usuario.
Recoge todos los datos introducidos en el formulario cuando se pulsa
el botón de enviar. Comprueba que ninguno de ellos es vacío o
incorrecto, y en caso de que todos sean correctos manda una petición
AJAX al script de PHP \texttt{sign\_up.php}, mandándole todos los
datos del formulario por POST. En caso de que algún dato esté mal,
bien porque no se ha introducido, bien porque haya algún error en el
mismo, el script modifica el CSS del elemento en cuestión, haciendo
que el borde se ponga rojo, y mostrando debajo un mensaje indicando el
error que se ha cometido. En caso de que haya algún error, la petición
AJAX indicada anteriormente no se realiza, si no que se deja al
usuario corregir los errores antes de enviarla, todo esto sin recargar
la página. Los errores que se han contemplado son los siguientes:

\begin{itemize}
\item Cualquiera de los campos obligatorios está vacío.
\item No se ha subido una foto.
\item La dirección de correo introducida no es una dirección de correo
  válida. Para esto, se ha tomado una expresión regular que comprueba
  si una cadena de texto tiene el formato de una dirección de correo
  electrónico (una cadena de caracteres seguida de una \@ y seguida de
  otra cadena de caracteres en la que al menos hay un punto)
\item La contraseña introducida es demasiado corta. Se considera que
  una contraseña con menos de 8 caracteres es corta, y no se acepta en
  el sistema.
\item Las dos contraseñas indtroducidas no coinciden. Se pide, por
  seguridad, que el usuario introduzca dos veces la contraseña. Si las
  dos contraseñas que ha introducido no coinciden, se considera que el
  usuario se ha equivocado y se le pide que las vuelva a introducir.
\item Los teléfonos móvil y fijo introducidos no son válidos. Se
  considera que un teléfono es válido si se corresponde con una cadena
  de caracteres compuesta por un signo + al princpio (opcional)
  seguido de dígitos numéricos. De nuevo se usa una expresión regular
  para esta comprobación.
\end{itemize}

Pasamos ahora a comentar el script \texttt{updateUser.js}. Este script
es muy similar al anterior, hace las mismas comprobaciones, y si todo
es correcto envía una petición AJAX al archivo
\texttt{update\_user.php}.  De nuevo, si hay algún error, tiene el
mismo comportamiento que el script
anterior.\\

Pasamos ahora a comentar los scripts que gestionan el comportamiento
del foro. Aquí tenemos dos archivos para gestionar formularios y uno
para gestionar el evento de pasar el ratón por encima de la imagen de
un usuario en el foro. Comenzamos por la gestión de formularios.
Tenemos por un lado el script \texttt{createThread.js}, que gestiona
la creación de un hilo nuevo. Este archivo recoge el título y la
descripción del mismo que se introducen en el formulario cuando se
envía el mismo, comprueba que ninguno de los dos es vacío, y en caso
de ser ambos correctos manda una petición AJAX al servidor, al archivo
\texttt{create-thread.php}, para guardar ese hilo en la base de datos.
El otro script que tenemos para gestionar formularios del foro es
\texttt{createResponse.js}. Este archivo se encarga de manejar los
formularios que permiten introducir respuestas a los hilos del foro.
En este archivo hay una pequeña diferencia que no teníamos en todos
los demás archivos. Para la gestión de los hilos, el identificador del
hilo se manda por GET en lugar de por POST, es decir, en la URL en
cuestión. Esto se hace para que los enlaces que identifican hilos en
la página web puedan ser enviados por métodos externos y aún así esté
perfectamente identificado el hilo al que nos referimos. Por ejemplo,
se podría querer mandar por correo electrónico la URL que muestra la
información relativa al hilo con identificador $n$. En este caso, la
URL en cuestión sería
\texttt{betatun.ugr.es/\textasciitilde x31008316/centrodeportivoII/hilo.php?thread\_id=n}.

\section{Bibliografía}

\printbibliography

\end{document}
