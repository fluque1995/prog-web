\documentclass[11pt]{article}
\renewcommand{\baselinestretch}{1.05}
\usepackage[spanish]{babel}
\usepackage[utf8]{inputenc}
\usepackage{lipsum}

\usepackage{amsmath,amsthm,verbatim,amssymb,amsfonts,amscd}
\usepackage{graphicx, wrapfig}
\usepackage{float}
\usepackage{caption, subcaption}
\usepackage{tkz-fct}
\usetikzlibrary{babel}
\usepackage{pgfplots}
\usepackage{enumitem}
\usepackage{multicol, vwcol}
\usepackage{listingsutf8}
\usepackage{color}
\usepackage{hyperref}
\usepackage{booktabs}
\usepackage[sorting=none]{biblatex}
\bibliography{bibliography.bib}

\definecolor{lightgray}{gray}{0.95}

\hypersetup{
    bookmarks=true,         % show bookmarks bar?
    unicode=false,          % non-Latin characters in Acrobat’s bookmarks
    pdftoolbar=true,        % show Acrobat’s toolbar?
    pdfmenubar=true,        % show Acrobat’s menu?
    pdffitwindow=false,     % window fit to page when opened
    pdfstartview={FitH},    % fits the width of the page to the window
    pdftitle={Programación web - Práctica 2},    % title
    pdfauthor={Francisco Luque},     % author
    pdfsubject={Programación web},   % subject of the document
    pdfnewwindow=true,      % links in new PDF window
    colorlinks=true,        % false: boxed links; true: colored links
    linkcolor=red,          % color of internal links (change box color with linkbordercolor)
    citecolor=cyan,         % color of links to bibliography
    filecolor=magenta,      % color of file links
    urlcolor=blue           % color of external links
}

\setlength{\parindent}{0pt}
\topmargin0.0cm
\headheight0.0cm
\headsep0.0cm
\oddsidemargin0.0cm
\textheight23.0cm
\textwidth16.5cm
\footskip1.0cm
\theoremstyle{plain}

\newtheorem{theorem}{Teorema}
\newtheorem{corollary}{Corolario}
\newtheorem{lemma}{Lema}
\newtheorem{proposition}{Proposición}
\theoremstyle{definition}
\newtheorem{definition}{Definición}
\newtheorem{example}{Ejemplo}

\newcommand{\N}{\mathbb{N}}
\newcommand{\Z}{\mathbb{Z}}
\newcommand{\Q}{\mathbb{Q}}
\newcommand{\C}{\mathbb{C}}
\newcommand{\R}{\mathbb{R}}

\begin{document}

\title{Programación Web - Práctica 2\\
  DGIIM \\
  \large Web con contenido dinámico usando Javascript y PHP}
\author{Francisco Luque Sánchez}
\maketitle

\begin{figure}[H]
  \centering
  \includegraphics[width=.5\textwidth]{js_php.png}
\end{figure}


\section{Introducción}

En esta práctica se ha dotado de contenido dinámico a la página web
que se desarrolló en la práctica anterior, utilizando para ello PHP
como lenguaje de \textit{scripting} en la lado del servidor, y
JavaScript (concretamente la librería JQuery) como lenguaje
de \textit{scripting} en el lado del cliente.\\

Pasamos a comentar el desarrollo completo en más profundidad.

\section{Código en PHP de lado del servidor}

El código desarrollado para dotar a la página web de contenido
dinámico del lado del servidor se ha hecho utilizando PHP. Con este
lenguaje se gestionan todas las consultas a la base de datos, así como
la inclusión de contenido redundante en todos los documentos HTML que
conforman la página. Todos los archivos PHP con código PHP puro que se
han desarrollado se encuentran dentro de la carpeta
\texttt{php-includes}. Comentaremos brevemente cómo se han
estructurado los \textit{scripts} PHP dependiendo de la función que
realizan.

\subsection{Archivos PHP para incluir contenido redundante}

Como hemos dicho en la introducción, se ha usado este lenguaje para
incluir en los documentos HTML el código redundante que aparecía en
todos ellos. Los \textit{scripts} que se han desarrollado destinados a
esta funcionalidad son los siguientes:

\begin{itemize}
\item \texttt{generic-head.inc.php}: Este archivo se encarga de
  imprimir una cabecera genérica para todos los archivos HTML. En
  todos los documentos HTML que conforman la página se incluían al
  principio la codificación que utilizaba la página (UTF-8), el nombre
  de la misma, el autor, y los archivos CSS que especificaban el
  estilo que tenían el menú superior y el pie de página. Para evitar
  tener que modificar esta información en todos ellos cada vez que se
  produjese un cambio, se ha creado este script que introduce toda
  esta información en dichos documentos. Además, se utiliza este
  script para introducir el script de JavaScript que define JQuery, ya
  que todas las páginas con contenido dinámico lo necesitan. Los
  archivos CSS específicos de cada página, no obstante, se incorporan
  al código HTML de la misma manera que se hacía en la práctica 1.
\item \texttt{header.inc.php}: Este archivo se encarga de imprimir
  todo el código que se incluye en el menú superior de las páginas. De
  la misma forma que en caso anterior, todas las páginas cuentan con
  el menú superior en el que aparece el logo del gimnasio, el nombre
  del mismo, y el formulario de inicio de sesión. Para poder tener
  este menú junto con su funcionalidad en todas las páginas, se
  imprime toda la información necesaria para que funcione el mismo
  desde este archivo, tanto el código HTML como el script de
  JavaScript que gestiona el formulario de login.
\item\texttt{top-menu.inc.php}: Este archivo imprime todo el código
  HTML que define el menú de navegación que aparece en todas las
  páginas del sitio excepto la página principal (\texttt{index.php}).
\item \texttt{footer.inc.php}: Este archivo imprime el pie de página
  de la misma forma que lo hacían los anteriores. La única diferencia
  que tiene este archivo con el resto es que en lugar de imprimir
  directamente el código HTML, tiene una función que recibe un
  parámetro, correspondiente a la clase que se asigna al footer. Se ha
  hecho así porque no todas las páginas tienen el mismo footer, si no
  que éste tiene un comportamiento distinto en las páginas cuyo
  contenido ocupa más que la ventana y en las que esto no ocurre. De
  esta forma, con una misma función hemos conseguido manejar estos
  dos comportamientos, simplemente con el parámetro de la función que
  imprime el contenido del footer.
\end{itemize}

\subsection{Archivos PHP para el manejo de la base de datos}

El resto de archivos desarrollados se encargan del manejo de la base
de datos en el servidor. Se ha intentado estructurar el código en
clases, cada una de ellas representando cada una de las tablas de la
base de datos que se han creado para la práctica. Vamos
a describir más en profundidad dichas clases.\\

Comenzamos hablando del archivo \texttt{db-config.inc.php}. Este
archivo no implementa ninguna funcionalidad, sino que define las
constantes necesarias para el funcionamiento del resto de clases.  En
este archivo se incluye, por tanto, una constante que contiene la
información necesaria para la conexión a la base de datos (DSN), el
nombre de usuario y la contraseña del usuario de la base de datos, y
constantes con el nombre de las tablas creadas
(users, forum\_threads y forum\_responses).\\

El siguiente archivo a mencionar es
\texttt{database-management.inc.php}.  Este archivo define una clase
abstracta que representa una tabla en la base de datos. Los métodos
implementados en esta clase son los siguientes:

\begin{itemize}
\item \texttt{\_\_construct}: Este método es el constructor de la clase.
  Recibe como parámetro un array y rellena otro array (atributo de la
  clase) con la información contenida en el array que se pasa como
  argumento.
\item \texttt{getValue}: Esta función permite recuperar contenido del
  array atributo de la clase, especificando el nombre que lo identifica.
\item \texttt{connect}: Este método permite abrir una conexión con la
  base de datos.
\item \texttt{disconnect}: Este método cierra una conexión activa con
  la base de datos.
\end{itemize}

El resto de clases que se describirán a continuación heredan de la
clase comentada. Comentaremos además el resto de métodos que se
han añadido a cada una.\\

Comenzamos comentando la clase \texttt{User}, la cual se encuentra
implementada en el archivo \texttt{user.inc.php}. Esta clase representa
un usuario almacenado en la base de datos del sistema. Por este motivo,
su array de datos está preparado para contener toda la información
que se guarda de un usuario. Tiene implementados los siguientes métodos:

\begin{itemize}
\item \texttt{saveUser}: Este método guarda el usuario que llama al
  mismo en la base de datos. Abre una conexión utilizando la función
  de la clase que hereda, realiza la inserción, y cierra la conexión
  al finalizar. Esta función se utiliza a la hora de dar de alta
  a un usuario nuevo.
\item \texttt{logUser}: Este método intenta permitir el acceso a un
  usuario al sistema. Toma el nombre de usuario y la contraseña del
  array de datos del que dispone, y trata de buscar en la base de
  datos un usuario para el que coincida esa información. Si lo
  encuentra, inicia una sesión de PHP y añade al array de sesión
  una entrada nueva con el nombre de usuario. Si no encuentra
  al usuario, devuelve un mensaje diciendo que no ha sido posible
  identificar al usuario en la base de datos. Esta función se
  utiliza cuando se intenta acceder al sistema con el menú de
  inicio de sesión que aparece en la parte superior de todas
  las páginas.
\item \texttt{getUser}: Esta función de clase devuelve un usuario
  almacenado en la base de datos dado su nombre de usuario como
  argumento. Se utiliza para recuperar la información de un usuario
  cuando se accede a la pestaña de perfil.
\item \texttt{updateUser}: Esta función actualiza la información de
  un usuario en la base de datos cuando se guarda su perfil tras
  modificarlo. El funcionamiento es muy similar al de \texttt{saveUser},
  con la salvedad de que ahora la sentencia es de tipo \texttt{UPDATE},
  en lugar de \texttt{INSERT}.
\end{itemize}

Pasamos ahora a comentar
\section{Código en JavaScript de lado del cliente}

Pasamos a ver los archivos CSS que se han utilizado en la página. Se
han creado dos conjuntos de archivos distintos, unos que se cargan
cuando la página web se está consultando desde un ordenador (la
anchura de la página es mayor a 768 píxeles) y otros que se cargan
cuando se accede a la página desde ún móvil (la anchura del
dispositivo es menor a 768 píxeles). Para un desarrollo realmente
completo se habría tenido que especificar la configuración para otros
tamaños, pero dado que lo que se pretende con la práctica es aprender
el funcionamiento básico de CSS no se ha profundizado más en este
apartado. En realidad, si lo que se busca es crear un página
completamente responsive, crear el CSS desde cero es una tarea muy
costosa, y probablemente, teniendo en cuenta la cantidad de
dispositivos distintos que existen actualmente, cada uno con sus
dimensiones particulares, sería prácticamente imposible que la
visualización fuera perfecta en todos ellos. Usualmente, lo que se
hace en estos casos es utilizar librerías ya creadas que se preocupan
en que el diseño de las páginas web sea completamente responsive. De
esta forma, el desarrollador web no tiene que preocuparse de que la
página se visualice correctamente en todos los dispositivos, sino que
es la propia librería la que le aporta las herramientas para
conseguirlo de forma cómoda. Un ejemplo de estas librerías,
probablemente la más extendida, es \textit{Bootstrap}
\cite{bootstrap}, desarrollada por el equipo de \textit{Twitter}.

Veamos a continuación los diferentes archivos CSS que se utilizan en
la página web:

\begin{itemize}
\item \texttt{topbar.css}: En este CSS se establece el estilo de la
  barra superior de la página, en la que se muestra el logo de la
  misma, el nombre del gimnasio, y la información sobre la
  autenticación del usuario.
\item \texttt{top-menu.css}: En este CSS se establece el estilo del
  menú superior, que aparece en todas las páginas excepto en la
  principal, y es la que nos permite movernos por todo el sitio
  web. Una particularidad de este menú es que se ha conseguido que al
  descender por la página, el mismo ocupe siempre la parte superior,
  de forma que siempre sea visible.  Así, aunque un usuario se
  encuentre en la parte inferior de una página, podrá dirigirse a otra
  página sin tener que volver a subir. Esto se consigue con una
  combinación de la propiedad \texttt{position: sticky}, que permite
  desplazar un elemento con el resto de la página cuando al hacer
  \texttt{scroll} debería quedarse oculto, y la propiedad
  \texttt{z-index}, que permite que un elemento quede por encima del
  resto cuando se superponen.
\item \texttt{footer.css}: En este CSS se establece el estilo del pie
  de página. Este componente se puede observar en la parte inferior de
  todas las páginas. Se han desarrollado dos clases distintas para el
  pie. Una de ellas se mantiene en la parte inferior de la página
  independientemete del tamaño que ocupe el contenido de la misma, que
  se usa en las páginas en las que el contenido no rellena
  completamente la pantalla, como la página de inicio o la página de
  actividades. La otra coloca el footer debajo de todo el
  contenido. Esto se ha hecho para que, en las páginas que hay poco
  contenido, se rellene la pantalla completamente, a fin de no romper
  la estética de la misma.
\item Archivos \texttt{nombre-pagina.css}: Por cada página HTML que
  hay creada, se tiene además un archivo CSS en el que se especifica
  el estilo del contenido que hay en la página. Se ha optado por
  organizar el CSS de esta forma porque cada página tiene un contenido
  específico y distinto de las demás. De esta manera, la estructura de
  archivos permite encontrar el CSS referente a cada página de forma
  más sencilla, por si hay que realizar alguna modificación. Se podría
  haber optado, en algunos casos, por utilizar la misma hoja de
  estilos para más de una página, como es el caso de las páginas
  \texttt{servicios.html} y \texttt{tecnicos.html}, en los que la
  estructura que siguen ambas es muy similar. No obstante, se ha
  optado finalmente por la organización separada, para poder controlar
  de forma independiente ambas páginas en caso de que fuera necesario.
\end{itemize}

Los archivos indicados anteriormente se encuentran en el directorio
\texttt{static/css}. Además de estos archivos, dentro de este
directorio se encuentra la carpeta \texttt{mobile}. En ella se
encuentran los mismos archivos indicados anteriormente, pero en la
versión móvil. De nuevo, se podría haber tenido un solo archivo con el
CSS para cada página, y que las \textit{media queries} se realizasen
dentro del propio CSS, pero buscando una organización más modular del
código, dicha \textit{media query} se hace en el HTML, y se carga un
CSS u otro dependiendo del tamaño del dispositivo que acceda a la
página.\\

En la versión móvil, lo que se ha intentado, es colocar los elementos
que aparecen alineados horizontalmente en la versión de escritorio
colocados en vertical. Esto se ha hecho porque, usualmente, los usuarios
que acceden a una web desde un móvil lo hacen con el móvil colocado
en vertical, al contrario que pasa con las pantallas de escritorio,
que tienen una disposición apaisada. De esta forma, en la versión
de escritorio en la pantalla de actividades aparece el listado de
actividades a la izquierda y el de noticias a la derecha, mientras
que en la versión móvil aparecen arriba las actividades y debajo las
noticias.

\section{Aspectos de implementación}

El desarrollo de la práctica se ha realizado utilizando las
herramientas de desarrollador que aporta \textit{Google Chrome}. Estas
herramientas permiten cambiar, de forma cómoda, el tamaño de la
pantalla en la que se mostraría la página. Esto permite visualizar de
forma sencilla la página en distintos dispositivos, sin necesidad de
tener los mismos físicamente. Es con esta herramienta con la que se ha
trabajado para tratar de hacer una página \textit{responsive}.  Se ha
probado la misma en todas las opciones que el navegador incluye por
defecto, y se ha comprobado que la página se comporta bien en todas
ellas. De esta forma podemos garantizar, de forma más o menos segura,
que vamos a poder visualizarla correctamente en cualquier dispositivo.
No obstante, la visualización perfecta en todos ellos es difícil de
garantizar completamente al desarrollar el CSS completamente desde
cero.\\

En cuanto al comportamiento responsive de la página, se ha seguido una
filosofía similar a la que sigue \textit{Bootstrap} en su desarrollo.
Lo que se hace es tener en cuenta el ancho de la pantalla para
redimensionar los elementos, mientras que en vertical se mantienen
los tamaños independientemente de la altura del dispositivo. Esto se
hace porque suele ser más natural el \textit{scroll} vertical que
el horizontal.

\section{Uso de imágenes en la práctica}

El logotipo que se utiliza en la cabecera está sacado de una página
que distribuye imágenes vectoriales de uso comercial llamada
\textit{Vecteezy} \cite{vecteezy}. La foto que aparece en la página
principal fue subida por Justyn Warner a \textit{Unsplash} \cite{unsplash}.\\

El resto de imágenes están sacadas todas de dos páginas distinas, las
cuales son \textit{Pixabay} \cite{pixabay} y \textit{Pexels}
\cite{pexels}. Las tres páginas citadas ofrecen todas las imágenes con
una licencia \textit{Creative Commons 0}, que permite el uso de las
mismas de forma completamente libre, tanto para uso comercial como no
comercial, sin necesidad de citar a los autores de las mismas en el
trabajo final. No obstante, se ha considerado correcto citar las
fuentes por si fuera necesario comprobar la procedencia de las mismas.

\section{Bibliografía}

\printbibliography

\end{document}
